%!TEX root = ../Demo.tex
% 中文摘要
\begin{abstract}

数字电子技术和信息技术日新月异,在多媒体应用技术、计算机应用、图像
与语音信号处理,通信服务等众多领域中,都已经广泛地使用到基于数字信号处
理的理论与技术。快速傅里叶变换(FFT)在很大程度上推动了数字信号处理技术的
发展,因为它的运算时间和直接计算离散傅里叶变换(DFT)相比大大减少,更有
利于实现数字信号处理的应用,并为数字信号处理技术在实时处理各种信号的应
用中提供了优势。因此,本文对 FFT 算法和实现方法的研究对其进一步发展起到
了非常重要的理论指导和实践意义



    本文的研究内容立足于FPGA实现DIT-FFT算法,这种算法不仅计算流程图比较简单,
    而且具有结构良好的特点,更重要的是这中算法在理论上是最快的DFT算法,只需要
    $N /2\log_2 N$ 次复数乘法运算。本文首先对DIT-FFT算法进行
    简单的理论说明及其推到,并分模块具体分析了每个模块应该实现
    哪些功能。最后,用Verilog硬件描述语言对各个共功能模块进行功能仿真
    ,仿真实现16点复数的FFT。用测试信号对本设计进行测试,并分析
    计算结果和Matlab计算结果的误差,结果表明本文的设计正确。

\end{abstract}
\keywords{FFT变换, DIT-FFT算法, Verilog HDL,蝶形算法}

% 英文摘要
\begin{enabstract}
  This paper is just a sample example for the users in learning the \XDUthesis. I will try my best to use the commands and environments which are involved by the \XDUthesis. Also, the popular composition skills in figures, tables and equations will be elaborated.
  
  In the part unimportant, I will show something others, such as poems and lyrics.

\end{enabstract}
\enkeywords{XDUthesis, commands, environments, skills}
